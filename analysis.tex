\section{Analyse des donn\'ees}
Cette manipulation a pour but la caract\'erisation d'un module optique (OM), ainsi que la pr\'eparation du dispositif n\'ecessaire. Cela implique :

\begin{center}
\fbox{
\begin{minipage}{0.75\textwidth}
Dans un premier temps, de se familiariser avec le dispositif :
\begin{itemize}
\item \'etudier l'efficacit\'e des PMs,
\item calibrer l'ADC,
\item d\'evelopper la logique d'aquisition de donn\'ees.
\end{itemize}

Ensuite, en se basant sur les donn\'ees, de calculer :
\begin{itemize}
\item le gain $G$ de l'OM,
\item la r\'esolution $\sigma_\mathrm{G}$ de l'OM,
\item le nombre moyen de photo-\'electrons $\langle n_{\mathrm{pe}}\rangle$ produit par trigger dans l'OM.
\end{itemize}
\end{minipage}
}
\end{center}
\pagebreak
